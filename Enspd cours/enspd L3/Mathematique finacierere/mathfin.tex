\documentclass{beamer}
\usetheme{Madrid}
\usecolortheme{seahorse}
\usepackage{graphicx}
\usepackage{xcolor}
\usepackage{booktabs}

% Couleurs personnalisées
\definecolor{maincolor}{RGB}{0, 102, 204}
\definecolor{accentcolor}{RGB}{255, 140, 0}

% Styles de titre
\setbeamercolor{title}{fg=white, bg=maincolor}
\setbeamercolor{frametitle}{fg=white, bg=maincolor}
\setbeamercolor{block title}{bg=accentcolor, fg=white}
\setbeamercolor{block body}{bg=white, fg=black}

\title[Formules Mathématiques Financières]{\textbf{Formules Essentielles de Mathématiques Financières}}
\author{Géniel Gandaho}
\institute{Université de Parakou \\ École Nationale de Statistique, de Planification et de Démographie}
\date{\today}

\begin{document}

% Page de titre
\begin{frame}
    \titlepage
\end{frame}

% Plan de la présentation
\begin{frame}{Plan de la présentation}
    \tableofcontents
\end{frame}

% Section : Intérêt Simple
\section{Intérêt Simple}
\begin{frame}{Formules d'Intérêt Simple}
    \begin{block}{Formules principales}
        \textbf{Intérêt total :}
        \[
        I = C \cdot t \cdot T
        \]
        \textbf{Valeur acquise (capital final) :}
        \[
        VA = C + I = C \cdot (1 + t \cdot T)
        \]
    \end{block}
    \begin{block}{Cas spécifiques}
        \begin{itemize}
            \item Durée en mois :
            \[
            I = \frac{C \cdot t \cdot m}{1200}
            \]
            \item Durée en jours :
            \[
            I = \frac{C \cdot t \cdot n}{36000}
            \]
        \end{itemize}
    \end{block}
\end{frame}

% Section : Intérêt Composé
\section{Intérêt Composé}
\begin{frame}{Formules d'Intérêt Composé}
    \begin{block}{Formules principales}
        \textbf{Valeur acquise :}
        \[
        VA = C \cdot (1 + t)^T
        \]
        \textbf{Valeur actuelle (capital initial) :}
        \[
        C = \frac{VA}{(1 + t)^T}
        \]
        \textbf{Taux équivalent :}
        \[
        1 + t_1 = (1 + t_2)^n
        \]
    \end{block}
\end{frame}

% Section : Intérêt Continu
\section{Intérêt Continu}
\begin{frame}{Formules d'Intérêt Continu}
    \begin{block}{Formules principales}
        \textbf{Valeur acquise :}
        \[
        VA = C \cdot e^{t \cdot T}
        \]
        \textbf{Valeur actuelle :}
        \[
        C = \frac{VA}{e^{t \cdot T}}
        \]
    \end{block}
\end{frame}

% Section : Escompte
\section{Escompte}
\begin{frame}{Formules d'Escompte Simple}
    \begin{block}{Formules principales}
        \textbf{Montant de l'escompte :}
        \[
        E = C \cdot t \cdot T
        \]
        \textbf{Valeur actuelle :}
        \[
        VA = C - E = C \cdot (1 - t \cdot T)
        \]
    \end{block}
\end{frame}

% Section : Annuités
\section{Annuités}
\begin{frame}{Formules des Annuités}
    \begin{block}{Valeur actuelle des annuités constantes}
        \[
        VA = R \cdot \frac{1 - (1 + t)^{-n}}{t}
        \]
    \end{block}
    \begin{block}{Valeur acquise des annuités constantes}
        \[
        VA = R \cdot \frac{(1 + t)^n - 1}{t}
        \]
    \end{block}
\end{frame}

% Section : Évaluation des Investissements
\section{Évaluation des Investissements}
\begin{frame}{Formules d'Évaluation des Investissements}
    \begin{block}{Valeur actuelle nette (VAN)}
        \[
        VAN = \sum_{t=0}^n \frac{CF_t}{(1 + r)^t} - C_0
        \]
    \end{block}
    \begin{block}{Taux interne de rendement (TIR)}
        \[
        \sum_{t=0}^n \frac{CF_t}{(1 + TIR)^t} = C_0
        \]
    \end{block}
\end{frame}

% Examen Final
\section{Examen Final}
\begin{frame}{Exercice 1 : Intérêt Simple (5 points)}
Un capital de \( 25,000 \, \text{F} \) est placé à un taux annuel de \( 8\% \) pendant \( 4 \, \text{ans} \).
\begin{enumerate}
    \item Calculez l’intérêt produit.
    \item Déterminez la valeur acquise.
\end{enumerate}
\end{frame}

\begin{frame}{Exercice 2 : Intérêt Composé (5 points)}
Un montant de \( 15,000 \, \text{F} \) est placé à un taux d’intérêt composé annuel de \( 5\% \). Après combien d’années le capital atteindra-t-il \( 20,000 \, \text{F} \) ?
\end{frame}

\begin{frame}{Exercice 3 : Annuités Constantes (6 points)}
Une entreprise rembourse \( 10,000 \, \text{F} \) chaque année pendant \( 7 \, \text{ans} \), à un taux d’intérêt annuel de \( 6\% \).
\begin{enumerate}
    \item Calculez la valeur actuelle des paiements.
    \item Déterminez la valeur acquise.
\end{enumerate}
\end{frame}

\begin{frame}{Corrigé Type}
\textbf{Exercice 1 : Intérêt Simple} \\
\[
I = C \cdot t \cdot T = 25,000 \cdot 0.08 \cdot 4 = 8,000 \, \text{F}
\]
\[
VA = C + I = 25,000 + 8,000 = 33,000 \, \text{F}
\]
\end{frame}

\begin{frame}{Corrigé Type (Suite)}
\textbf{Exercice 2 : Intérêt Composé} \\
\[
VA = C \cdot (1 + t)^T \implies 20,000 = 15,000 \cdot (1.05)^T
\]
\[
T = \frac{\ln(1.3333)}{\ln(1.05)} \approx 5.89 \, \text{ans}
\]
\end{frame}

\begin{frame}{Merci !}
    \centering
    \Huge Des questions ?
\end{frame}

\end{document}
